
\newcount\columnsperpage
\overfullrule=0pt

\columnsperpage=3
\def\versionnumber{0.99}  % Version of this reference card
\def\year{2006}
\def\month{Apr}
\def\version{\month\ \year\ v\versionnumber}

\def\shortcopyrightnotice{\vskip .5ex plus 2 fill
  \centerline{\small \copyright\ \year\ v\versionnumber}}

\def\copyrightnotice{\vskip 1ex plus 100 fill\begingroup\small
\centerline{\version.}
\endgroup}

\def\bye{\par\vfill\supereject\end}

\newdimen\intercolumnskip
\newbox\columna
\newbox\columnb

\def\ncolumns{\the\columnsperpage}

\message{[\ncolumns\space 
  column\if 1\ncolumns\else s\fi\space per page]}

\def\scaledmag#1{ scaled \magstep #1}

\if 1\ncolumns
  \hsize 4in
  \vsize 10in
  \voffset -.7in
  \font\titlefont=\fontname\tenbf \scaledmag3
  \font\headingfont=\fontname\tenbf \scaledmag2
                \font\headingfonttt=\fontname\tentt \scaledmag2
  \font\smallfont=\fontname\sevenrm
  \font\smallsy=\fontname\sevensy

  \footline{\hss\folio\hss}
  \def\makefootline{\baselineskip10pt\hsize4in\line{\the\footline}}
\else
  \hsize 3.2in
  \vsize 7.5in %% was 7.95in
  \hoffset -2mm % was -.75in
  \voffset -.745in
  \font\titlefont=cmbx10 \scaledmag2
  \font\headingfont=cmbx10 \scaledmag1
  \font\headingfonttt=cmtt10 \scaledmag1
  \font\smallfont=cmr6
  \font\smallsy=cmsy6
  \font\eightrm=cmr8
  \font\eightbf=cmbx8
  \font\eightit=cmti8
  \font\eighttt=cmtt8
  \font\eightsy=cmsy8
  \font\eightsl=cmsl8
  \font\eighti=cmmi8
  \font\eightex=cmex10 at 8pt
  \textfont0=\eightrm  
  \textfont1=\eighti
  \textfont2=\eightsy
  \textfont3=\eightex
  \def\rm{\fam0 \eightrm}
  \def\bf{\eightbf}
  \def\it{\eightit}
  \def\tt{\eighttt}
  \def\sl{\eightsl}
  \normalbaselineskip=.8\normalbaselineskip
  \normallineskip=.8\normallineskip
  \normallineskiplimit=.8\normallineskiplimit
  \normalbaselines\rm           %make definitions take effect

  \if 2\ncolumns
    \let\maxcolumn=b
    \footline{\hss\rm\folio\hss}
    \def\makefootline{\vskip 2in \hsize=6.86in\line{\the\footline}}
  \else \if 3\ncolumns
    \let\maxcolumn=c
    \nopagenumbers
  \else
    \errhelp{You must set \columnsperpage equal to 1, 2, or 3.}
    \errmessage{Illegal number of columns per page}
  \fi\fi

  \intercolumnskip=.46in
  \def\abc{a}
  \output={%
      % This next line is useful when designing the layout.
      %\immediate\write16{Column \folio\abc\space starts with \firstmark}
      \if \maxcolumn\abc \multicolumnformat \global\def\abc{a}
      \else\if a\abc
        \global\setbox\columna\columnbox \global\def\abc{b}
        %% in case we never use \columnb (two-column mode)
        \global\setbox\columnb\hbox to -\intercolumnskip{}
      \else
        \global\setbox\columnb\columnbox \global\def\abc{c}\fi\fi}
  \def\multicolumnformat{\shipout\vbox{\makeheadline
      \hbox{\box\columna\hskip\intercolumnskip
        \box\columnb\hskip\intercolumnskip\columnbox}
      \makefootline}\advancepageno}
  \def\columnbox{\leftline{\pagebody}}

  \def\bye{\par\vfill\supereject
    \if a\abc \else\null\vfill\eject\fi
    \if a\abc \else\null\vfill\eject\fi
    \end}  
\fi

% we won't be using math mode much, so redefine some of the characters
% we might want to talk about
%\catcode`\^=12
%\catcode`\_=12
\catcode`\~=12

\chardef\\=`\\
\chardef\{=`\{
\chardef\}=`\}
\chardef\underscore=`\_
\chardef\'="0D % These are upright quote marks


\hyphenation{}

\parindent 0pt
\parskip .85ex plus .35ex minus .5ex

\def\small{\smallfont\textfont2=\smallsy\baselineskip=.8\baselineskip}

\outer\def\newcolumn{\vfill\eject}

\outer\def\title#1{{\titlefont\centerline{#1}}\vskip 1ex plus .5ex}

\outer\def\section#1{\par\filbreak
  \vskip .5ex  minus .1ex {\headingfont #1}\mark{#1}%
  \vskip .3ex  minus .1ex}

\outer\def\librarysection#1#2{\par\filbreak
  \vskip .5ex  minus .1ex {\headingfont #1}\quad{\headingfonttt<#2>}\mark{#1}%
  \vskip .3ex  minus .1ex}


\newdimen\keyindent

\def\beginindentedkeys{\keyindent=1em}
\def\endindentedkeys{\keyindent=0em}
\def\begindoubleindentedkeys{\keyindent=2em}
\def\enddoubleindentedkeys{\keyindent=1em}
\endindentedkeys

\def\paralign{\vskip\parskip\halign}

\def\<#1>{$\langle${\rm #1}$\rangle$}

\def\kbd#1{{\tt#1}\null}        %\null so not an abbrev even if period follows

\def\beginexample{\par\vskip1\jot
\hrule width.5\hsize
\vskip1\jot
\begingroup\parindent=2em
  \obeylines\obeyspaces\parskip0pt\tt}
{\obeyspaces\global\let =\ }
\def\endexample{\endgroup}

\def\Example{\qquad{\sl Example\/}.\enspace\ignorespaces}

\def\key#1#2{\leavevmode\hbox to \hsize{\vtop
  {\hsize=.75\hsize\rightskip=1em
  \hskip\keyindent\relax#1}\kbd{#2}\hfil}}

\newbox\metaxbox
\setbox\metaxbox\hbox{\kbd{M-x }}
\newdimen\metaxwidth
\metaxwidth=\wd\metaxbox

\def\metax#1#2{\leavevmode\hbox to \hsize{\hbox to .75\hsize
  {\hskip\keyindent\relax#1\hfil}%
  \hskip -\metaxwidth minus 1fil
  \kbd{#2}\hfil}}

\def\threecol#1#2#3{\hskip\keyindent\relax#1\hfil&\kbd{#2}\quad
  &\kbd{#3}\quad\cr}


\title{Emacs Python mode Reference Card}

\section{Misc commands}

\metax{C-j:} {Insert a new line with the same indentation level }{as the current line}

\metax{RET:}{ Insert a new line with the same indentation }{level as the current line}

\metax{C-M-a:}{ Go to the beginning of the current}{ function or class}

\metax{C-M-e:}{ Go to the end of the current }{function or class}

\metax{C-M-h:}{ Mark the current function or class}{ for copying, etc.}

\metax{C-M-x:}{ Execute the current function or class}

\metax{C-c C-b:}{ Submit a bug report}

\metax{C-c C-c:}{ Execute the buffer (i.e., the file}{ being displayed)}

\metax{C-c C-d:}{ Trace the stack of the process being executed}

\metax{C-c C-h:}{ Get context-based help}

\metax{C-c TAB:}{ Indent a highlighted (or marked) region}

\metax{C-c C-k:}{ Mark a block of text. Using this}{ at the head of a class or function definition}{ will mark the entire block.}

\metax{C-c C-l:}{ Shift the region to the left. If}{ the cursor is in the middle of a region,}{ the lower half of the region will shift.}

\metax{C-c RET:}{ Execute the current file, opening}{ a new window to show the output.}

\metax{C-c C-n:}{ Jump to the next statement.}

\metax{C-c C-p:}{ Jump to the previous statement.}

\metax{C-c C-r:}{ Shift the region to the right. If}{ the cursor is in the middle of a region,}{ the lower half of the region will shift.}

\metax{C-c C-s:}{ Execute a Python command.}

\metax{C-c C-t:}{ Toggle shells}

\metax{C-c C-u:}{ Go up one block}

\metax{C-c C-v:}{ List the version of the Python mode}

\metax{C-c C-w:}{ Run PyChecker}

\metax{C-c !:}{ Open the Python interactive shell}

\metax{C-c \#:}{ Comment the highlighted (marked) region}

\metax{C-c :}{:}{ Check the indentation off-set}

\metax{C-c <:}{ Shift the region to the left}

\metax{C-c >:}{ Shift the region to the right}

\metax{C-c ?:}{ Show Python mode documentation}

\metax{C-c |:}{ Execute the highlighted (marked) part}{ of the current program. }


\bye


